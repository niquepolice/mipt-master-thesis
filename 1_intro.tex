\section*{Введение}
\addcontentsline{toc}{section}{\protect\numberline{}Введение}

\paragraph{Актуальность темы.}
Магистерская работа посвящена исследованию задачи распределённого управления изолированной энерго(электро)системой.
Для поддержания устойчивости энергосистемы и эффективности её работы в изменяющихся условиях (например, изменение уровня потребления электроэнергии) необходимо в реальном времени решать задачу выбора мощности генерации (\textit{уставок}) для каждой электростанции.
Поскольку электрическая сеть~--- физически связанная система, уставки для генерирующих электростанций должны согласовываться между собой.

Традиционно (например, в Единой энергетической системе России \cite{eesfreqpower}), эта задача решается централизованным/иерархическим способом: диспетчерские центры собирают информацию о состоянии энергосистемы и выдают уставки генерирующим станциям.
В современных энергетических системах с внедрением возобновляемых источников энергии (ВИЭ) увеличивается количество генерирующих устройств, генерация распределяется по потребительской сети.
Домашние хозяйства с установленными солнечными и ветроэнергетическими станциями могут в разные моменты времени выступать как потребители, или как производители электроэнергии.
Увеличение количества генераторов и динамически меняющаяся топология сети создаёт проблемы для централизованных систем управления: проходящие через один узел потоки данных перегружают каналы передачи информации, вычислительная сложность задачи может превышать аппаратные возможности одного устройства.
С распространением дешёвых и производительных одноплатных компьютеров построение надёжной (устойчивой к отказу одного узла) и быстрой системы управления на основе децентрализованного подхода может стать более эффективным решением, чем реализация традиционной централизованной схемы.

Другая особенность современных энергосистем, связанная с внедрением ВИЭ~--- влияние внешних условий на генерацию электроэнергии.
Суточные осцилляции солнечной активности и стохастическая природа ветра приводят к рассогласованию между спросом и предложением и возникновению избыточной электроэнергии, особенно в изолированных системах с большой долей ВИЭ.
Для решения этой проблемы приходится использовать накопители энергии и/или подстраивать график потребления электроэнергии под профиль генерации ВИЭ.

Учет этих особенностей приводит к тому, что постановка задачи управления энергосистемой существенно отличается от традиционной.

В России применение технологий микрогридов и ВИЭ особенно актуально в северных районах, изолированных от единой энергосети.
Используемые в изолированных поселениях на данный момент дизельные электростанции экологически и экономически неэффективны~--- экономически обоснованные тарифы на электроэнергию могут быть в десятки раз выше, чем в присоединённых к единой энергосети областях \cite[15]{ancenter2017tarifes}.
Так, в 2019--2021 годах Инжиниринговый центр «Арктическая Автономная Энергетика» МФТИ выполнял проект модернизации действующей дизельной генерации с введением в эксплуатацию ветроэнергетической установки в энергоизолированном посёлке Лаборовая Приуральского района Ямало-Ненецкого автономного округа.
Математическое и программное обеспечение этого проекта~--- одна из целей выполнения данной работы.

В данный момент темы распределённых информационных систем и распределённых алгоритмов оптимизации характеризуются высокой научной активностью.
Это же свойственно и теме \textit{микрогридов}~--- единиц энергетической сети с распределённой генерацией \cite{lasseter2002microgrids}.
% краткий обзор?
Поэтому имеется очень много результатов о распределённых системах оптимизации и обосновании оптимизационных постановок задач управления изолированными энергетическими системами, однако, насколько нам известно, комплексного описания распределённой системы управления для изолированной энергосистемы с различными типами генераторных и аккумулирующих устройств и преобладанием тепловой нагрузки (что характерно для объектов, расположенных на северных территориях) в литературе нет
%(конечно, надежда на то, что какой-нибудь индус не опубликовал аналогичную статью лет 10 назад довольна слаба, но зачем её искать и портить себе настроение). 


This is not surprising as OPF has been
shown in [25], [39], and [40] to be NP-hard in general. \cite{lehmann2015ac}

 
 
 \paragraph{Цель работы.}
 Целью работы является разработка децентрализованной системы управления изолированной гетерогенной по типу систем генерации и накопления энергии энергосистемой, учитывающей преобладающий характер тепловой нагрузки для применения в проектируемых и перспективных объектах арктической зоны.
 
 \paragraph{Методы исследования.}
 Для достижения поставленных целей используется аппарат теории оптимизации.
 Программная реализация выполнена на языке Python3.
 Проверка работы системы осуществлена на тестовом стенде Инжинирингового центра «Арктическая Автономная Энергетика» МФТИ в составе кластера одноплатных компьютеров Raspbery PI и Tinkerboard, на которых размещена система управления, и программно-аппаратного комплекса физического моделирования режима реального времени RTDS.
 
 
 
 \paragraph{Основные положения, выносимые на защиту.}
 \begin{enumerate}
   \item Децентрализованная система управления изолированным микрогридом, разработанная с учетом особенностей арктического применения.
   \item Методы анализа критериев эффективности системы управления, относящихся к использованию систем накопления энергии.
 \end{enumerate}
 
 
 
 \paragraph{Научная новизна.}
Получена верхняя оценка эффективности системы накопления энергии, в которой одно устройство  работает в астатическом по частоте и напряжению режиме, а остальные~--- в астатическом по мощности режиме.
Проведён анализ алгоритмов подсчёта циклов в кривых заряда-разряда устройств накопления энергии.
Дано комплексное описание принципов построения децентрализованной системы управления изолированным микрогридом с уклоном в применение в арктических условиях.
 
 
%  \paragraph{Теоретическая значимость.}
%  Данное исследование вносит существенный вклад в область согласования прогнозов иерархических временных рядов, связанный с разработкой нового подхода к согласованию прогнозов, который не имеет недостатков, присущих ранее разработанным методам, и, следовательно, имеет более широкую область применения.
 
 
 \paragraph{Практическая значимость.}
 Разработанная система отличается простотой, эффективностью и широким потенциалом применения в активно развивающейся области модернизации и введения в эксплуатацию новых арктических автономных систем энергоснабжения.
 
 
 \paragraph{Степень достоверности и апробация работы.}
 Достоверность результатов подтверждена математическими доказательствами и экспериментальной проверкой работы системы на детальной физической модели. 
 Результаты работы использовались в коммерческих проектах Инжинирингового центра «Арктическая Автономная Энергетика» МФТИ.
 
 
%  \paragraph{Публикации по теме.}
%  Основные результаты по теме магистерской работы изложены в изданиях из списка ВАК \cite{medvednikova2012algorithm, kuznetsov2012algorithm, medvednikova2013construction, stenina2014reconciliation, gazizullina2015forecasting, stenina2015reconciliation}, двух сборниках докладов конференций \cite{stenina2014mipt, stenina2015lomonosov} и других печатных изданиях \cite{medvednikova2012pca, Stenina2015ordinal}.
 
 