\section{Централизованный подход}

\subsection{Постановка задачи}
\label{sec:planning_problem}
\subsubsection{Модель}

Решается задача оптимального управления автономной энергетической системой.\\
Рассматриваемая система включает неконтроллируемый источник ``бесплатной'' возобновляемой энергии  (ветрогенератор), дизель-генератор, литий-ионную аккумуляторную батарею и постоянную неуправляемую нагрузку.
Требуется найти оптимальные с точки зрения экономического критерия уставки мощности для дизель-генератора и батареи с условием поддержания баланса мощности (здесь и далее \textit{мощность} означает \textit{активная мощность}) в системе в течении следующего после принятия решения часа \textit{(задача планирования)}.
На ближайший час значение мощности ветрогенерации считается постоянным и известным.

Используется линейная модель дизель-генератора и батареи, близкая к модели, принятой в пакете \textbf{Homer Pro}, часто используемом для валидации в литературе по планированию в микрогридах \cite{Berendes2018, Aziz2019, Petersen2018, Olatomiwa2016}. 
Соответствующие модели графики изображены на Рис \ref{fig:bgcost}.

Для включенного дизельного генератора зависимость расхода топлива от мощности:
\begin{equation}
 \text{fuelConsumption} (P) = \text{fuelIntercept} + \text{fuelSlope} \cdot P.
\end{equation}

Тогда стоимость использования дизель-генератора:

\begin{equation}\label{f:cg}
c_g(P) \left[\frac{\$}{h} \right] = b_g + a_g \cdot P,
\end{equation}


\begin{equation}\label{f:bg}
b_g \left[\frac{\$}{h} \right] = 
\text{fuelIntercept} \cdot \text{fuelCost} + \text{maintainanceCost} +
\frac{\text{gensetPrice}}{\text{gensetResource}},
\end{equation}

\begin{equation}\label{f:ag}
a_g \left[ \frac{\$}{kW\cdot h} \right] = 
\text{fuelSlope} \cdot \text{fuelCost}.
\end{equation}

Здесь maintainanceCost~[\$ / моточас]~--- стоимость обслуживания,\\ 
gensetPrice~[\$]~--- капитальная стоимость генераторной установки,\\
gensetResource~[моточас]~--- количество моточасов до замены.

Пренебрегая зависимостью интенсивности износа аккумуляторной батареи от глубины заряда/разряда и величины токов, считаем, что стоимость использования батареи определяется её капитальной стоимостью и ресурсом, равным количеству электрической энергии, которая может пройти через батарею до того, как батарею необходимо будет заменить.
Таким образом, стоимость использования батареи определяется по формуле \cite{bwc}:


\begin{equation}\label{f:cb}
c_b(P) \left[\frac{\$}{h} \right] = 
a_b \cdot P,
\end{equation}
\begin{equation}\label{f:ab}
a_b \left[ \frac{\$}{kW\cdot h} \right] =
\frac{\text{batteryPrice}}{\text{batteryCapacity} \cdot \text{batteryMaxCycles} 
\cdot \sqrt{\eta}} ,
\end{equation}

здесь batteryPrice~[\$]~--- капитальная стоимость батареи, batteryCapacity~[$kW\cdot h$]~--- энергетическая ёмкость батареи, batteryMaxCycles~--- число полных циклов заряда-разряда до замены батареи, $\eta$~--- энергетический КПД цикла заряда-разряда.

\subsubsection{Стратегии}

Простейшим подходом к задаче поиска экономически оптимального на всём периоде эксплуатации системы управления с соблюдением баланса мощности при отсутствии информации о будущей ветрогенерации является применение эвристических правил-стратегий.

В пакете \textbf{Homer Pro}, по умолчанию используются две такие стратегии: \textit{Load Following (LF)} \cite{lf} и \textit{Cycle Charging (CC)} \cite{cc}.

В стратегии LF уставка генератору выбирается такой, чтобы обеспечить баланс мощности, не производя дополнительной энергии для зарядки батареи или, при наличии, обслуживания откладываемой нагрузки (defferable load).
То есть согласно стратегии LF выбирается управление, наиболее дешёвым образом образом обеспечивающее баланс мощности на ближайший квант времени планирования (локально-оптимальное).

Выбор управления по стратегии СС состоит из двух этапов: сначала находится управление по стратегии LF, после чего для генераторов с ненулевыми уставками уставки повышаются до номинальной мощности генераторов или до максимальной мощности, которую готовы принять аккумуляторная батарея и откладываемая нагрузка.

Также в литературе \cite[10]{Aziz2019} встречается несколько иная формулировка стратегии CC, будем обозначать её CC2. 
Здесь, как и в стратегии LF ищется локально-оптимальное управление, но в экономический критерий оптимальности вводится дополнительная величина, \textit{стоимость энергии батареи} \cite{bec}:
\begin{equation}\label{f:bec}
    a_{be} = \frac{C_{cc}}{E_{bc}},
\end{equation}

где $C_{cc}$ --- расходы на энергию, произведённую специально для того, чтобы зарядить батарею (в процессе Cycle Charging); $E_{bc}$ --- суммарное количество энергии, переданное батарее.
И в ограничения добавляется условие, что, как и в стратегии $CC$ мощность генератора должна быть либо нулевой, либо быть равной минимуму из его номинальной мощности и максимальной мощности, которую готова принять система.

Зарядка батареи от ветрогенератора считается бесплатной, поэтому в числителе учитываются только расходы от использования дизель-генератора для зарядки батареи.
При оптимизации считается, что при разрядке батареи система теряет деньги, равные стоимости энергии батареи, умноженной на энергию разряда, а при зарядке - получает.
То есть если батарея часто заряжается от ветрогенератора, стоимость энергии батареи мала и системе не выгодно заряжать батарею от генератора, а разряжать батарею, наоборот, не затратно.
Если же ветрогенерации мало, стоимость энергии батареи высока и системе выгодно увеличивать мощность дизель-генератора для зарядки батареи.

Для корректной работы алгоритма необходимо разумно инициализировать стоимость энергии батареи, пока батарея не получала энергию и пользоваться формулой (\ref{f:bec}) невозможно.
До этого момента полагаем $a_{be} = a_b / 2$.

\subsubsection{MILP}
Теперь сформулируем задачи поиска управления по стратегиям LF, CC, CC2 в виде задач смешанного целочисленно-линейного программирования~(MILP).

\paragraph{LF}
\label{sub:lf}
\begin{equation}\label{f:lf}
\begin{split}
\vspace{-4mm}
&f = -\eps z_+ - a_b z_- 
+ b_g x + a_g y
\rightarrow \min\limits_{(x, y, z_+, z_-) \in \{0,1\} \times \R^3 }\\
\text{s.t. }& y + P_{wind} \geq P_{load} + z_+ + z_-\\
&y \leq x \cdot P_{genset~max}\\
&0 \leq y \\
&0 \leq z_+ \leq P_{battery~ch~max} \\
&-P_{battery~disch~max} \leq z_- \leq 0 \\
\end{split}
\end{equation}
Здесь $x$ --- включён ли генератор,\\
$y$ --- мощность генератора,\\
$z_+$ --- мощность заряда батареи,\\
$z_-$ --- мощность разряда батареи со знаком минус,\\
$\eps$ --- достаточно малое число --- поощрение за заряд батареи,\\
$P_{battery~ch~max},~P_{battery~disch~max}$~--- максимальные доступные на следующий час мощности заряда и разряда батареи, посчитанные с учётом её текущего заряда.

Уставки мощности генератору и батарее (разряду соответствует отрицательная уставка) равны $y$ и $z_+ + z_-$ соответственно.

Поскольку в стратегию $LF$ не заложена ценность энергии батареи, но $LF$ подразумевает, что вся избыточная энергия ветрогенерации по возможности поглащается батареей, в критерий оптимальности вводится дополнительное слагаемое $-\eps z_+$, а плата за износ батареи при её заряде переносится в плату за разряд батареи.


\paragraph{CC}
Пусть $(x, y, z_+, z_-)$ --- решение (\ref{f:lf}).\\
Пусть $P_{gm} := \min(P_{genset~max}, P_{load}- P_{wind} + P_{battery~ch~max})$.\\
Тогда уставки генератору и батарее равны соответственно $P_{gm}x$ и $z_+ + z_- + P_{gm}x - y$.

\paragraph{CC2}

\begin{equation}\label{f:cc2}
\begin{split}
&f = \frac{a_b}{2}(z_+ - z_-) - a_{be}(z_+ + z_-)
+ (b_g  + a_g P_{gm}) x
\rightarrow \min\limits_{(x, z_+, z_-) \in \{0,1\} \times \R^2 }\\
\text{s.t. }& y + P_{wind} \geq P_{load} + z_+ + z_-\\
&0 \leq z_+ \leq P_{battery~ch~max} \\
&-P_{battery~disch~max} \leq z_- \leq 0 \\
\end{split}
\end{equation}


Здесь плата за износ батареи разделена поровну между зарядом и разрядом.
Уставки генератора и батареи равны $P_{gm}x$ и $z_+ + z_-$.

\medskip
Для всех этих оптимизационных задач точное решение мгновенно вычисляется дефолтным солвером PuLP.

\subsection{Результаты моделирования}
Три описанные выше стратегии были использованы для управления системой на 134-часовом периоде и оценены по критериям суммарной экономической эффективности, количеству использованного топлива,  ресурса батареи и дизель-генератора.
Код симуляции доступен по \href{https://github.com/niquepolice/electro/blob/03f136b13663806bfba4a4ce973f524558564363/milp.ipynb}{ссылке}.

В качестве параметров системы были взяты значения:\\
$\text{gensetPrice} = \dfrac{2.5e6}{75}\$$\\
$\text{gensetResource} = 15000h$\\
$\text{maintainanceCost} = 3\$/h$\\
$\text{fuelIntercept} = 2L/h$\\
$\text{fuelSlope} = (28-2)/100 L/kWh$\\
$\text{fuelPrice} = \dfrac{64}{75}\$/L$\\
% $\text{batteryPrice} = \frac{64}{75}\$/L$
$\text{batteryMaxCycles} = 3000$\\
$\text{batteryCapacity} = 100kWh$\\
$\eta = 1$ \\
$P_{genset max} = 100kW$ \\
$P_{battery max} = 3\text{batteryCapacity} / h$

Для удельной (по ёмкости) стоимости батареи протестированы два значения, соответствующие двум разным режимам работы системы: когда $a_b > a_g$ ($1000\$/kWh$) и $a_b < a_g$ ($400\$/kWh$).
Графики стоимости использования батареи и генератора двух таких режимов показаны на Рис~\ref{fig:bgcost}.


\begin{figure}[h]
\includegraphics[scale=0.5]{energy_cost.jpeg}
\caption{Стоимость использования батареи и дизель-генератора}
\centering
\label{fig:bgcost}
\end{figure}

\medskip

Для нагрузки так же протестированы два значения: 70кВт и 30кВт.

Результаты представлены в Таблице \ref{t:res} и на Рисунках \ref{fig:res_70_400}, \ref{fig:res_70_1000} и \ref{fig:res_30} .
По результатам видно, что стратегии CC и СС2 незначительно отличаются при большой нагрузке и абсолютно эквивалентны при малой, поэтому для нагрузки 30кВт приведены только графики по CC. 
При дешёвой батарее лучшие результаты показывают CC и СС2, а при дорогой~--- LF, причем при малой нагрузке и дешёвой батарее разница существенна, а при остальных параметрах выражена менее явно.
Всех случаях LF экономнее по отношению к ресурсу батареи, а СС~--- по отношению к топливу и моточасам.


\begin{table}[h]
\caption{Результаты моделирования}
\label{t:res}
\begin{tabular}{lllllll}
strategy & load & battery cost & battery resource, \% & fuel consumption & motohours & total cost \\
lf       & 70   & 400          & 99.89                & 1667             & 100       & 2022       \\
cc       & 70   & 400          & 99.18                & 1567             & 56        & 1974       \\
cc2      & 70   & 400          & 99.55                & 1645             & 72        & 1985       \\
lf       & 70   & 1000         & 99.9                 & 1660             & 97        & 2058       \\
cc       & 70   & 1000         & 99.69                & 1654             & 79        & 2154       \\
cc2      & 70   & 1000         & 99.82                & 1714             & 89        & 2136       \\
lf       & 30   & 400          & 99.9                 & 549              & 61        & 849        \\
cc       & 30   & 400          & 99.55                & 447              & 16        & 650        \\
cc2      & 30   & 400          & 99.55                & 447              & 16        & 650        \\
lf       & 30   & 1000         & 99.9                 & 542              & 58        & 882        \\
cc       & 30   & 1000         & 99.55                & 447              & 16        & 918        \\
cc2      & 30   & 1000         & 99.55                & 447              & 16        & 918       
\end{tabular}
\end{table}

\subsection{Направления дальнейшей работы}

\begin{itemize}
    \item Расширение модели\\
    Формулировка задачи в общем случае --- для произвольного количества генераторов, батарей, электролизёров и топливных элементов, исследование вычислительной эффективности алгоритма для больших систем.

    \item Учет нелинейностей \\
    Учёт нелинейности топливной характеристики дизель-генератора,
    зависимости износа батареи от глубины и тока заряда/разряда.
    В таком случае могут быть вычислительные проблемы с поиском точного решения задачи оптимизации, применяются эвристические алгоритмы \cite{Sufyan2019}.
    
    \item Глобальная оптимальность\\
    Хотя в литературе есть результаты, показывающие, что простые стратегии управления в некоторых случаях могут давать глобально-оптимальные решения \cite{Barley1996}, представляет интерес поиск универсальных глобально-оптимальных алгоритмов управления.
    Другие подходы к поиску глобально-оптимальных решений, включающие другие эвристические стратегии и применение [средневековых] методов искусственного интеллекта, например, генетических алгоритмов и нечёткой логики, могут быть найдены в литературе \cite{Olatomiwa2016}.
    
    Один из подходов, в которых явно заложен учёт оптимальности работы системы на следующих итерациях --- использование модельно-прогностического контроля (Model Predictive Control, MPC).
    В статье \cite{Tazvinga2014} целевая функция MPC была составлена таким образом, чтобы минимизировать использование дизель-генератора и максимизировать использование ВИЭ; 
    было показано, что при наличии ошибок в прогнозировании потребления и генерации ВИЭ управление с помощью MPC на тестовой модели эффективнее управления с помощью квадратичного программирования (аналогично сформулированной выше задаче MILP для LF (\ref{f:lf}), только функция расхода дизеля имеет вид $aP^2 + bP$).
    
    
    В общем виде формулировка задачи представляется таковой:
    \begin{equation}
        \E\left( \sum_{k=0}^m c_k ~\vert~ u_0 \right) \rightarrow \min\limits_{u_0},
    \end{equation}
    
    где $c_k$ --- затраты на $k$-ом интервале планирования, случайная величина, зависящая от вероятностных распределений случайных процессов ветрогенерации и нагрузки; $u_0$~--- управление на ближайший интервал.
    
    Частными случаями такого подхода являются стратегии $CC$ и $LF$ --- при ветрогенерации равной нулю или достаточно большой константе соответственно.
    
    
    \item Оперативная коррекция\\
    Учёт возможности изменения нагрузки и ветрогенерации после планирования. 
    Простейшее решение --- добавить в задачу оптимизации условие на наличие резерва мощности, доступного без включения на этапе оперативной коррекции новых генераторов.
    При анализе решений задачи планирования для оценки эффективности по этому параметру (\textit{reliability analysis}) оценивают вероятность нехватки мощности \cite[8]{Sufyan2019} или вносят условие на её максимальное значение в ограничения оптимизационной задачи \cite[5]{Petersen2018}.
    
    \item Баланс реактивной мощности\\
    Есть публикации \cite{zhang2016reactive}.
    
\end{itemize}

\begin{figure}[]
\includegraphics[scale=0.45]{/lf_70_400.png}
\includegraphics[scale=0.45]{/cc_70_400.png}
\includegraphics[scale=0.45]{/cc2_70_400.png}
\centering
\caption{Распределение нагрузки 70кВт, 1000\$/кВт ч}
\label{fig:res_70_400}
\end{figure}

\begin{figure}[]
\includegraphics[scale=0.45]{/lf_70_1000.png}
\includegraphics[scale=0.45]{/cc_70_1000.png}
\includegraphics[scale=0.45]{/cc2_70_1000.png}
\caption{Распределение нагрузки 70кВт, 400\$/кВт ч}
\label{fig:res_70_1000}
\end{figure}

\begin{figure}[]
\includegraphics[scale=0.45]{/lf_30_400.png}
\includegraphics[scale=0.45]{/cc_30_400.png}
\includegraphics[scale=0.45]{/lf_30_1000.png}
\includegraphics[scale=0.45]{/cc_30_1000.png}
\caption{Распределение нагрузки 30кВт}
\label{fig:res_30}
\end{figure}


\section{Планирование и коррекция}
\subsection{Обновления постановки задачи}
    Двигаясь дальше, мы уточняем модель и постановку задачи для большего соответствия процессам в реальной электросистеме.
    Основные изменения по сравнению с предыдущей моделью относятся к учёту различия между прогнозируемыми и наблюдаемыми средними значениями мощности ветрогенерации и учёту отклонений мгновенных значений от средних. 
    
    Теперь управление разделяется на два этапа: 
    \begin{enumerate}
        \item  Планирование.
        \item Оперативная коррекция.
    \end{enumerate}
    
    \subsubsection{Планирование}
        На этом этапе в соответствии с данным в разделе \label{sec:planning_problem} описанием задачи происходит планирование режима работы на следующий час.
        Принципиальной ролью этого этапа является выбор оборудования, которое будет подготовлено к запуску и включено на весь следующий этап планирования, т.к. некоторые виды генераторов, систем аккумуляции и откладываемой нагрузки требуют время на подготовку к включению и выходу на рабочий режим и/или имеют ограничения на частоту включения/выключения, связанные с повышенным их износом при этих операциях.
        Таким элементам соответствуют дискретные (бинарные) переменные оптимизационной задачи.
        На данный момент сюда входит только дизельный генератор, в скором времени в модель также добавится система генерации и накопления водорода и протонно-обменный топливный элемент.
        
        Второй (незадействованной на данный момент) функцией этого этапа является выдача постоянных референсных уставок оборудованию на весь период планирования, роль которых для получения глобально-оптимального управления будет обсуждена ниже.
        
        Этап планирования реализован по стратегии LF, так же, как описано в предыдущем разделе, но с двумя отличиями: 
        \begin{enumerate}
            \item  вместо реальной средней мощности ветрогенерации на следующий период используется \textit{наивное предсказание} равное среднему за предыдущий период,
            
            \item  в оптимизационную задачу \ref{f:lf} добавлено мягкое ограничение 
            maxGeneration - load~$\geq$ reserve 
            на обеспечение запаса мощности для компенсации колебаний ветрогенерации.
            Технически добавлена вспомогательная переменная $t$, одно ограничение типа неравенства и слагаемое в целевой функции:
            \begin{equation}
                \begin{split}
                    x \cdot P_{genset~max} + P_{wind} - P_{battery~disch~max} + t &\geq P_{load} + P_{reserve}\\
                    \Tilde{f} &= f + I \cdot t, ~ t \geq 0,
                \end{split}
            \end{equation}
        \end{enumerate}
        где $I$ --- достаточно большое число.
        
    \subsubsection{Оперативная коррекция}
        Для компенсации кратковременных колебаний ветрогенерации в модель добавляется ещё одна \textit{опорно-балансирующая (опорная)} батарея.
        Она управляется локальной автоматикой и принимает на себя все отклонения от заложенного на этапе оперативной коррекции прогноза.
        Оперативная коррекция также реализована на основе LF, но с некоторыми модификациями.
        
        Во-первых, штрафуется изменение каждой бинарной переменной(относительно её запланированного/текущего значения): 
        на этапе коррекции в штатной ситуации не должно происходить включение оборудования, которому соответствуют дискретные переменные.
        Переменные остаются переменными для случая резкого отклонения реальных погодных условий от прогнозированных. 
        Вероятно, это должно быть заменено внеплановым вызовом алгоритма планирования.
        Если бинарная переменная была равна 1 к началу очередной итерации коррекции, то она считается константой (в задаче нет жёстких ограничений, которые могли бы заставить платить штраф за смену значения дискретной переменной с 1 на 0).
        В с случае с одной дискретной переменной её значение может быть определено до решения оптимизационной задаче и в случае, когда она была равна 0 к началу итерации (из условия возможности соблюдения баланса мощности.
        
        
        \textcolor{red}{П-регулятор для батареи не нужен. 
        При кпд батареи = 1 при отсутствии регулятора энергия батареи будет колебаться в диапазоне шириной $P_{wind_max} \cdot \text{correctionTimeStep}$, повторяя профиль ветрогенерации (полностью поглощая колебания ветра). При кпд < 1 нужно лишь подзаряжать батарею примерно раз в день, когда кривая энергии батареи сползает вниз из-за потерь.
        При этом остаётся проблема с тем, что при обеспечении нагрузки от батареи гибкости (дизель выключен, ветер в среднем меньше нагрузки) колебания ветра приводят к тому, что заряд перетекает между батареей гибкости и обратно.
        Это неизбежно, если считать, что задача опорно-балансирующей батареи сделать так, как будто реальная ветрогенерация не отличается от той, что ожидалась при планировании(коррекции). 
        Расчёт ожидаемой ветрогенерации по среднему за несколько последних периодов вместо одного улучшит ситуацию, в случае, когда в профиле ветра есть колебания значительной амплитуды с периодом равным периоду итерации коррекции (что наблюдается на наших данных, но общность под вопросом). Актуальные графики см в 1-year-hysteresis-dahsboard.ipynb в ветке milp}
        
        Во-вторых, в целях поддержания заряда опорной батареи на уровне 50\%, значение нагрузки корректируется соответственно его отклонению.
        Точнее говоря, к величине нагрузки (наивному предсказанию) на входе оптимизационной задачи добавляется выход пропорционального регулятора: 
        \begin{equation}
        \label{f:dL}
        \Delta L := \alpha \frac{0.5 \cdot  \text{batteryMaxCharge} -
        \text{batteryCharge}}{\text{correctionTimeStep}},
        \end{equation}
        
        где $\alpha$ --- настраиваемый коэффициент регулятора.
        На тестовых данных его оптимальное значение (по критерию минимизации максимального отклонения заряда опорной батареи от 50\%) равно около $0.6$.
        
        Также были попытки добавить в задачу штраф за отклонение от референсных уставок, полученных на этапе планирования.
        Допустим, получение мощности от батареи существенно дешевле получения мощности от дизель-генератора. Допустим также, что заряда батареи хватит чтобы покрыть ровно половину потребности в энергии на следующий период планирования.
        В таком случае батарее и дизель-генератору будут выданы одинаковые уставки мощности, равные
        $L/2 = \text{batteryCharge} / \text{planningTimeStep}$.
        Допустим, что погода соответсвует прогнозу, и ветрогенерация на всем рассматриваемом периоде планирования достаточно близка к своему среднему.
        Тогда локально-оптимальный алгоритм коррекции в течении первой половины периода для батареи и генератора будет выбирать уставки $(L, 0)$, а вторую половину, когда батарея уже разрядится, $(0, L)$.
        Может показаться, что это плохо, но в случае линейных функций стоимости итоговые затраты за период не меняются. 
        Зато принуждение алгоритма коррекции к следованию референсным уставкам приводит к повышению затрат при существенном отклонении погоды от прогноза. 
        Поэтому на текущий момент референсные уставки с этапа планирования при коррекции не используются, но описанное в этом примере свойство поведения локально-оптимальных алгоритмов коррекции нужно иметь ввиду при переходе к нелинейным моделям.
        
        Для решения задачи не нужно иметь входные данные о нагрузке и ветрогенерации.
        Поскольку используется наивное предсказание, то нужны только данные о нагрузке и ветрогенерации, а точнее об их разности, за предыдущий период.
        Предполагая, что оборудование на предыдущем периоде строго следовало уставкам, получаем, что достаточно знать изменение  уровня заряда опорной батареи в начале данного и предыдущего периода планирования/коррекции 
        $\Delta E_{core~battery, i}$:
        
        \begin{equation}
        \label{f:dP}
        \Delta P := \Delta P_i =  
        P_{wind, i-1} - P_{load, i-1} 
        = -y_{i-1}  + (z_{+, i-1} + z_{-, i-1})
        + \frac{\Delta E_{core~battery, i}}{\Delta t} 
        \end{equation}
        
        C учётом \ref{f:dL} и \ref{f:dP} математическая формулировка задачи операционной коррекции имеет вид (используется обозначения из \ref{sub:lf}):
 \begin{equation}\label{f:lf}
\begin{split}
\vspace{-4mm}
&f =  -a_b z_- + b_g x + a_g y + I(r_+ + r_-)
\rightarrow \min\limits_{(y, z_+, z_-, r_+, r_-) \in \R^5 }\\
\text{s.t. }& \Delta P - \Delta L + y = z_+ + z_- + r_+ - r_- \\
&y \leq x \cdot P_{genset~max}\\
&0 \leq y \\
&0 \leq r_+,~r_- \\
&0 \leq z_+ \leq P_{battery~ch~max} \\
&-P_{battery~disch~max} \leq z_- \leq 0 \\
\end{split}
\end{equation} 

Здесь условие сохранения баланса мощности является мягким ограничением типа равенства.
Это обусловлено желанием точно следовать выработанному плану контроля уровня заряда опорной батареи и необходимо (?) при применении описанных ниже модификаций целевой функции, а также позволяет избавиться от фиктивной награды за заряд батареи.

Из логики, что желательно поддерживать уровень заряда управляемой батареи в некотором диапазоне около 50\%, было выдвинуто предложение
ввести квадратичную зависимость стоимости нахождения управляемой батареи в данном состоянии от уровня заряда в этом состоянии (квадратичный потенциал с минимумом в 50\%).
С точки зрения задачи линейного программирования это означает переменный (по абсолютному значению и знаку) коэффициент стоимости использования батареи (парабола эффективно линеаризуется на одноминутных интервалах коррекции):
\begin{equation}
    a'_b = a_b \cdot 2 \left(\frac{\text{batteryCharge}}{\text{batteryMaxCharge}} - 0.5\right) \cdot k,
\end{equation}

$k$ --- настраиваемый параметр.

Отсюда две альтернативные целевые функции, соответствующие чисто потенциальному подходу и смешанному с LF:

\begin{equation}
\begin{split}
f_q &= (a'_b + \eps) z_+ + (a'_b - \eps) z_- + b_g x + a_g y + I(r_+ + r_-)\\
f_m &= (a'_b + \eps) z_+ + (a'_b - \eps - a_b) z_- + b_g x + a_g y + I(r_+ + r_-)
\end{split}
\end{equation}
         
\subsection{Результаты моделирования}
        
    Для численного моделирования использовались интерполированные до одной минуты данные по ветрогенерации в Тикси.
    
    Программная реализация на PuLP показала неудовлетворительное для проведения исследовательской работы время исполнения,
    вероятно, по причине того, что решение оптимизационных задач внутри цикла на каждой итерации требовало от операционной системы создания нового процесса
    (большинство практически применимых солверов в PuLP --- бинарники с интерфейсом командной строки).
    Поэтому оптимизиционные задачи были переписаны в формате входных параметров для функции linprog модуля scipy.optimize.
    На данный момент scipy не поддерживает дискретные переменные в задачах линейного программирования, поэтому на этапа планирования для выбора переменной включения дизель-генератора решались две оптимизационные задачи, в каждой из которых эта переменная считалась константой, а на этапе коррекции её значение определялось предварительно из условия соблюдения баланса мощности.
    Это дало ускорение вычислений в $>7$ раз.
    
    Результаты моделирования для батареи стоимостью 1000\$ / кВтЧ и 400\$ / кВтЧ показаны на рисунках \ref{fig:corr-1000} и \ref{fig:corr-400} соответственно.
    Для LF-функции потерь $a'_b = k a_b$.
    
    Как видно по результатам, в данных конфигурациях применение потенциального подхода не даёт выигрыша.
    
    На рисунке \ref{fig:core} показан типичный график изменения заряда опорной батареи.
    
    Ipynb с симуляцией и интерактивными графиками доступен по  \href{https://github.com/niquepolice/electro/blob/milp/milp.ipynb}{ссылке}.
    
    
    \begin{figure}[]
\includegraphics[scale=0.5]{/corr-1000d.png}
\centering
\caption{Зависимость итоговой стоимости от коэффициента k, 1000\$/кВт ч}
\label{fig:corr-1000}
\end{figure}

\begin{figure}[]
\includegraphics[scale=0.5]{/corr-400d.png}
\centering
\caption{Зависимость итоговой стоимости от коэффициента k, 400\$/кВт ч}
\label{fig:corr-400}
\end{figure}

\begin{figure}[]
\includegraphics[scale=0.5]{/core.png}
\centering
\caption{График изменения заряда опорной батареи}
\label{fig:core}
\end{figure}

